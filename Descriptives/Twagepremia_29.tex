\begin{table}[ht]
\caption{Changes in wage premia for experience and educational attainment at age 29 for full-time workers}
\label{tab:wagepremia}
\centering
\scalebox{1.0}[1.0]{% 
\begin{threeparttable}
\begin{tabular}{lrrr@{}l}
\toprule 
Variable & NLSY79 & NLSY97 & \multicolumn{2}{c}{97--79} \\
\midrule 
\multicolumn{5}{l}{\emph{Average log wage premia for one more year of experience:}} \\
~~Year of School    & 0.066 & 0.056 & -0.010 &   * \\ 
~~Work in HS        & 0.055 & 0.031 & -0.024 &       \\ 
~~Work in college   & 0.089 & 0.067 & -0.022 & *** \\ 
%~~~~Early college work & 0.218 & 0.165 & -0.053 & * \\ 
%~~~~Late college work  & 0.099 & 0.079 & -0.020 & ** \\ 
~~~~Early college work & 0.137 & 0.065 & -0.072 & ** \\ 
~~~~Late college work  & 0.073 & 0.068 & -0.005 &   \\ 
~~Work part time    & -0.091 & -0.128 & -0.037 &  *** \\ 
~~Work full time    & 0.005 & 0.002 & -0.003 &    \\ 
\vspace{-6pt}  \\
\multicolumn{5}{l}{\emph{Average log wages by highest educational attainment:}} \\
~~HS Dropouts        & 1.81 & 1.75 & -0.05 &    \\ 
~~HS Graduates       & 1.95 & 1.92 & -0.03 &    \\ 
~~Some College       & 2.09 & 1.99 & -0.10 &  *** \\ 
~~College Graduates  & 2.35 & 2.30 & -0.05 &  * \\ 
\vspace{-6pt}  \\
\multicolumn{5}{l}{\emph{Average log wage premia for highest educational attainment:}} \\
~~High School Wage Premium   & 0.14 & 0.16 & 0.02 &    \\ 
~~Some College Wage Premium  & 0.14 & 0.07 & -0.07 &  *** \\ 
~~College Wage Premium       & 0.41 & 0.38 & -0.03 &   *** \\ 
\bottomrule 
\end{tabular} 
\footnotesize{Notes: The sample is conditional on working full-time. Estimates for work experience are coefficients from separate bivariate regressions of log wage on each cumulative experience term. The exception is for the breakout of Work in college. \emph{Early college work} refers to work done as a Freshman or Sophomore (no more than 16 months of college, ie 4 semesters), while \emph{Late college work} refers to work done as a Junior or Senior (more than 16 months of college). The premia for these two experiences are from a joint regression. \emph{HS Graduates} included in this table are those who never attended college. \emph{Some College} are those who attended college but did not graduate with a 4-year degree. \emph{College Graduates} are those who graduated with a 4-year degree. \emph{High School Wage Premium} refers to the log wage difference between \emph{HS Graduates} and \emph{HS Dropouts}. \emph{Some College Wage Premium} refers to the log wage difference between \emph{Some College} and \emph{HS Graduates}. \emph{College Wage Premium} refers to the log wage difference between \emph{College Graduates} and \emph{HS Graduates}. Statistics weighted by NLSY sampling weights. Significance reported at the 1\% (***), 5\% (**), and 10\% (*) levels.}
\end{threeparttable} 
} 
\end{table} 
