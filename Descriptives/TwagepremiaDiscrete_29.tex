\begin{table}[ht]
\caption{Changes in wage premia for experience and educational attainment at age 29 for full-time workers}
\label{tab:wagepremiaDiscrete}
\centering
\scalebox{1.0}[1.0]{% 
\begin{threeparttable}
\begin{tabular}{lrrr@{}l}
\toprule 
Variable & NLSY79 & NLSY97 & \multicolumn{2}{c}{97--79} \\
\midrule 
\multicolumn{5}{l}{\emph{Average log wage premia for one more year of experience:}} \\
~~Year of School    & 0.058 & 0.063 & 0.005 &    \\ 
~~Work in HS        & 0.092 & 0.075 & -0.017 &    \\ 
~~Work in college   & 0.239 & 0.056 & -0.183 &    \\ 
~~Work part time    & -0.186 & -0.175 & 0.011 &    \\ 
~~Work full time    & -0.080 & 0.032 & 0.112 &  *** \\ 
\end{tabular} 
\footnotesize{Notes: The sample is conditional on working full-time. Estimates for work experience are coefficients from separate bivariate regressions of log wage on each cumulative experience term. \emph{HS Graduates} included in this table are those who never attended college. \emph{Some College} are those who attended college but did not graduate with a 4-year degree. \emph{College Graduates} are those who graduated with a 4-year degree. \emph{High School Wage Premium} refers to the log wage difference between \emph{HS Graduates} and \emph{HS Dropouts}. \emph{Some College Wage Premium} refers to the log wage difference between \emph{Some College} and \emph{HS Graduates}. \emph{College Wage Premium} refers to the log wage difference between \emph{College Graduates} and \emph{HS Graduates}. Statistics weighted by NLSY sampling weights. Significance reported at the 1\% (***), 5\% (**), and 10\% (*) levels.}
\end{threeparttable} 
} 
\end{table} 
